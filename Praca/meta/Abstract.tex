\chapter*{Abstract \small{(author: Piotr Wesołowski)}}
\indent The engineering thesis presents a system for analyzing ECG signals using machine learning methods, focusing on heart rate variability (HRV) and the respiratory sinus arrhythmia (RSA) effect. The project is based on real-time data processing using the Polar H10 sensor and a dedicated application that enables the acquisition of ECG signals transmitted via Bluetooth. The system also supports the analysis of complete signal datasets loaded from files, allowing for comprehensive investigation of data dependencies.

The system implements various data analysis techniques, each requiring the prior identification of characteristic peaks in the ECG signal using a dedicated algorithm. Based on the detected peaks, intervals between them were calculated, forming the foundation for further analysis. The primary analysis, performed in the time domain, allowed for the calculation of several key metrics based on these intervals. These intervals were also utilized in frequency and time-frequency analyses. Advanced mathematical transformations, such as the Fast Fourier Transform and wavelet analysis, were applied in these methods. The results of the analyses related to heart rate variability are presented as numerical values of key metrics and their time-varying characteristics on clear visual graphs.

The thesis compares the effectiveness of a traditional algorithm for R-peak detection with a machine learning-based algorithm. The applied neural networks achieved high accuracy in detecting characteristic peaks in the ECG signal and predicting various heart rate variability metrics based on them. An innovative approach was developed, enabling the prediction of HRV and RSA parameters directly from the entire ECG signal, representing a significant extension of the traditional RR-interval-based approach.

The thesis experimentally examined the impact of breathing patterns on RSA and analyzed the diurnal variability of heart activity by comparing differences between daytime and nighttime. A qualitative analysis of the developed neural network models was also conducted, highlighting their strengths, potential development directions, and possible applications in biomedical data analysis.

The system and methods presented in this thesis constitute a comprehensive platform for ECG analysis, with potential applications in scientific research and medical diagnostics.
\vspace{0.5cm}\newline
\textbf{Keywords:} Machine learning, EKG, HRV, RSA, Polar H10, end-to-end analysis \vspace{0.5cm}