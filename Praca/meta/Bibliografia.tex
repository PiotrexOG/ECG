\bibliographystyle{plain}
\begin{thebibliography}{99}
    \addcontentsline{toc}{chapter}{Wykaz literatury}
    \small

    %------------------------------------------
    %Lilly, Leonard S. (2016). Pathophysiology of Heart Disease: A Collaborative Project of Medical Students and Faculty, 6th Edition. Lippincott Williams & Wilkins. pp. 70�78. ISBN 978-1-4698-9758-5. OCLC 1229852550.

    \bibitem{Lilly-Pathophysiology}
    Leonard S. Lilly \emph{Pathophysiology of Heart Disease: A Collaborative Project of Medical Students and Faculty, 6th Edition.} Lippincott Williams \& Wilkins. pp. 70�78. ISBN 978-1-4698-9758-5. OCLC 1229852550.
    %------------------------------------------

    %------------------------------------------
    \bibitem{Malik-HRV}
    Task Force of the European Society of Cardiology, \& North American Society of Pacing and Electrophysiology \emph{Heart rate variability: Standards of measurement, physiological interpretation, and clinical use.} In: Circulation. 1996, pp. 1043-1065. doi:10.1161/01.CIR.93.5.1043.
    %PDF
    %https://www.ahajournals.org/doi/10.1161/01.CIR.93.5.1043
    %https://www.researchgate.net/publication/279548912_Heart_rate_variability_Standards_of_measurement_physiological_interpretation_and_clinical_use
    %------------------------------------------

    %------------------------------------------
    \bibitem{Fariha-PanTompkins}
    M. A. Z. Fariha et al. \emph{Analysis of Pan-Tompkins Algorithm Performance with Noisy ECG Signals}, J. Phys.: Conf. Ser., vol. 1532, p. 012022, 2020. Available at: \href{https://iopscience.iop.org/article/10.1088/1742-6596/1532/1/012022/pdf}{https://iopscience.iop.org/article/10.1088/1742-6596/1532/1/012022/pdf.} [Accessed 08.11.2024]
    %https://iopscience.iop.org/article/10.1088/1742-6596/1532/1/012022/pdf}{https://iopscience.iop.org/article/10.1088/1742-6596/1532/1/012022/pdf
    %------------------------------------------

    \bibitem{Khan-PanTompkins++}
    Khan, Naimul; Imtiaz, Md Niaz (2023). Pan-Tompkins++: A Robust Approach to Detect R-peaks in ECG Signals. Toronto Metropolitan University. Preprint. https://doi.org/10.32920/22734308.v1

    %TO DO: dopasowa� strony glownie rozdzial 2
    \bibitem{Romano-ECG}
    M. Roman{\`o} and R. Bertona. \emph{Text Atlas of Practical Electrocardiography: A Basic Guide to ECG Interpretation.} Springer Milan, 2015. ISBN: 9788847057418. %Available at: \href{https://books.google.pl/books?id=qH8QBwAAQBAJ}{https://books.google.pl/books?id=qH8QBwAAQBAJ}. [Accessed 12.11.2024]

    % https://books.google.pl/books?hl=pl&lr=&id=qH8QBwAAQBAJ&oi=fnd&pg=PP17&dq=The+Practical+Guide+to+ECG+Interpretation&ots=xtUWsyms0z&sig=GFw3_NCeNPsIWaDzl1LhZNcJ4pY&redir_esc=y#v=onepage&q=The%20Practical%20Guide%20to%20ECG%20Interpretation&f=false
    %https://books.google.pl/books?id=qH8QBwAAQBAJ&printsec=frontcover&hl=pl&source=gbs_atb#v=onepage&q&f=false

    \bibitem{Goodfellow-DeepLearning}
    I. Goodfellow, Y. Bengio, and A. Courville, \emph{Deep Learning}. Cambridge, MA, USA: MIT Press, 2016. [Online]. Available: \url{http://www.deeplearningbook.org}[Accessed 25.11.2024]

    %------------------------------------------
    \bibitem{Bartsch-PhaseTransitions}
    R. P. Bartsch, A. Y. Schumann, J. W. Kantelhardt, T. Penzel, and P. Ch. Ivanov,
    \emph{Phase transitions in physiologic coupling},
    Proc. Natl. Acad. Sci. U. S. A., vol. 109, no. 26, pp. 10181�10186, Jun. 2012.
    Available at: \href{https://doi.org/10.1073/pnas.1204568109}{https://doi.org/10.1073/pnas.1204568109}. [Accessed 08.11.2024]
    %------------------------------------------

    \bibitem{MUzairZahid-UNET}
    M. U. Zahid, S. Kiranyaz, T. Ince, O. C. Devecioglu, M. E. H. Chowdhury, A. Khandakar, A. Tahir, and M. Gabbouj,
    \emph{Robust R-Peak Detection in Low-Quality Holter ECGs Using 1D Convolutional Neural Network},
    *IEEE Transactions on Biomedical Engineering*, vol. 69, no. 1, pp. 119--128, Jan. 2021.

    \bibitem{FDA_KardiaAI_2018}
    U.S. Food and Drug Administration, \emph{Kardia AI Clearance - K181823}, 2018. [Online]. Available: \url{https://www.accessdata.fda.gov/cdrh_docs/pdf18/K181823.pdf}. [Accessed: Dec. 1, 2024].

    % Bumgarner, J. M., Lambert, C. T., Hussein, A. A., et al. (2018). Smartwatch Algorithm for Automated Detection of Atrial Fibrillation. Journal of the American College of Cardiology, 71(21), 2381-2388. DOI: 10.1016/j.jacc.2018.03.003.
    \bibitem{Bumgarner2018}
    J.~M. Bumgarner, C.~T. Lambert, A.~A. Hussein, \textit{et al.}, ``Smartwatch Algorithm for Automated Detection of Atrial Fibrillation,'' \textit{Journal of the American College of Cardiology}, vol. 71, no. 21, pp. 2381--2388, 2018, doi: \url{10.1016/j.jacc.2018.03.003}.

    % https://pmc.ncbi.nlm.nih.gov/articles/PMC9971999/
    \bibitem{Raghunath2023}
    A. Raghunath, D.~D. Nguyen, M. Schram, D. Albert, S. Gollakota, L. Shapiro, and A.~R. Sridhar, ``Artificial intelligence-enabled mobile electrocardiograms for event prediction in paroxysmal atrial fibrillation,'' \textit{Cardiovascular Digital Health Journal}, vol. 4, no. 1, pp. 21--28, Jan. 2023, doi: \url{10.1016/j.cvdhj.2023.01.002}. PMID: 36865584; PMCID: PMC9971999.

    \bibitem{ribeiro2020automatic}
    Ribeiro, A.H., Ribeiro, M.H., Paix?o, G.M.M. \emph{et al.}, \emph{Automatic diagnosis of the 12-lead ECG using a deep neural network}, \emph{Nature Communications}, vol. 11, p. 1760, 2020, doi: 10.1038/s41467-020-15432-4.
    % Ribeiro, A.H., Ribeiro, M.H., Paix?o, G.M.M. et al. Automatic diagnosis of the 12-lead ECG using a deep neural network. Nat Commun 11, 1760 (2020). https://doi.org/10.1038/s41467-020-15432-4

    \bibitem{NAROTAMO-deepLearning}
    Deep learning for ECG classification: A comparative study of 1D and 2D representations and multimodal fusion approaches
    %     @article{NAROTAMO2024106141,
    % title = {Deep learning for ECG classification: A comparative study of 1D and 2D representations and multimodal fusion approaches},
    % journal = {Biomedical Signal Processing and Control},
    % volume = {93},
    % pages = {106141},
    % year = {2024},
    % issn = {1746-8094},
    % doi = {https://doi.org/10.1016/j.bspc.2024.106141},
    % url = {https://www.sciencedirect.com/science/article/pii/S174680942400199X},
    % author = {Hemaxi Narotamo and Mariana Dias and Ricardo Santos and Andr� V. Carreiro and Hugo Gamboa and Margarida Silveira},
    % keywords = {Electrocardiogram classification, Cardiovascular diseases, Deep learning, Recurrent neural networks, Convolutional neural networks, Multimodal artificial intelligence},
    % abstract = {The improved diagnosis of cardiovascular diseases (CVD) from electrocardiograms (ECG) may help prevent their severity. Since Deep Learning (DL) became popular, several DL methods have been developed for ECG classification. In this work, we compare how different methods for ECG signal representation perform in the multi-label classification of CVDs, including recent attention-based strategies. Furthermore, multimodal fusion strategies are employed to improve the prediction capacity of individual representation networks. The publicly available PTB-XL ECG dataset, which contains 21,837 records and labels for the diagnosis of 4 CVDs, was used for the task. Two DL strategies using different processing approaches were compared. Recurrent Neural Network-based models take advantage of the temporal dependence between raw signal values, namely through Gated Recurrent Unit (GRU), Long Short Term Memory (LSTM) and 1D-Convolutional Neural Network models. Additionally, the raw ECG was converted into image representations, based on recent work, and the classification was performed using distinct 2D-Convolutional Neural Networks. The potential of multimodal DL was then studied through early, late and joint data fusion strategies, to evaluate the benefit of resorting to multiple representations. Results based on the 1D ECG representation outperform image-based approaches and multimodal models. The best model, GRU, achieved sensitivity and specificity of 79.67% and 81.04%, respectively.}
    % }

    \bibitem{Morteza-RandomForest}
    Morteza Zabihi1*
    , Ali Bahrami Rad2*
    , Aggelos K. Katsaggelos3
    ,
    Serkan Kiranyaz4
    , Susanna Narkilahti2
    , and Moncef Gabbouj Detection of Atrial Fibrillation in ECG Hand-held Devices Using a Random
    Forest Classifier
    % https://physionet.org/files/challenge-2017/1.0.0/papers/069-336.pdf

    \bibitem{Bachmann-electrolyte_prediction}
    von Bachmann, P., Gedon, D., Gustafsson, F.K. et al. \emph{Evaluating regression and probabilistic methods for ECG-based electrolyte prediction.} Sci Rep 14, 15273 (2024). https://doi.org/10.1038/s41598-024-65223-w
    % von Bachmann, P., Gedon, D., Gustafsson, F.K. et al. Evaluating regression and probabilistic methods for ECG-based electrolyte prediction. Sci Rep 14, 15273 (2024). https://doi.org/10.1038/s41598-024-65223-w

    \bibitem{JUHO-LSTM}
    Laitala J., Jiang M., Haulivuori E., Kasaeyan Naeini E., Airola A., Rahmani A. M., Dutt N., Liljeberg P.:
    Robust ECG R-peak detection using LSTM, 2020, doi: 10.1145/3341105.3373945.
    %https://dl.acm.org/doi/pdf/10.1145/3341105.3373945

    \bibitem{PhysioNet-Challange2017}
    Clifford GD, Liu C, Moody B, Li-wei HL, Silva I, Li Q, Johnson AE, Mark RG. AF classification from a short single lead ECG recording: The PhysioNet/computing in cardiology challenge 2017. In 2017 Computing in Cardiology (CinC) 2017 Sep 24 (pp. 1-4). IEEE. https://doi.org/10.22489/CinC.2017.065-469

    \bibitem{Kavya-shockableArthytmiaDetection}
    Lakkakula Kavya, Karuna Yepuganti, Saladi Saritha, Allam Jaya Prakash, Kiran Kumar Patro, Suraj Prakash Sahoo, Ryszard Tadeusiewicz, Pawel Plawiak:
    A review of shockable arrhythmia detection of ECG signals using machine and deep learning techniques. Int. J. Appl. Math. Comput. Sci. 34
    % https://sciendo.com/article/10.61822/amcs-2024-0034

    \bibitem{Irungu-EcgCovid}
    J. Irungu, T. Oladunni, A. C. Grizzle, M. Denis, M. Savadkoohi, and E. Ososanya, "ML-ECG-COVID: A Machine Learning-Electrocardiogram Signal Processing Technique for COVID-19 Predictive Modeling," \textit{IEEE Access}, vol. 11, pp. 135994--136014, 2023, doi: 10.1109/ACCESS.2023.3335384.

    %     @ARTICLE{10325461,
    %   author={Irungu, John and Oladunni, Timothy and Grizzle, Andrew C. and Denis, Max and Savadkoohi, Marzieh and Ososanya, Esther},
    %   journal={IEEE Access}, 
    %   title={ML-ECG-COVID: A Machine Learning-Electrocardiogram Signal Processing Technique for COVID-19 Predictive Modeling}, 
    %   year={2023},
    %   volume={11},
    %   number={},
    %   pages={135994-136014},
    %   keywords={Electrocardiography;Feature extraction;Support vector machines;Machine learning;Classification algorithms;Time-frequency analysis;Random forests;Nearest neighbor methods;Support vector machine (SVM);random forest;QRS complex;K-nearest neighbor (KNN);electrocardiogram (ECG)},
    %   doi={10.1109/ACCESS.2023.3335384}}

    \bibitem{SAKR2023324}
    A.~S.~Sakr, P.~P�awiak, R.~Tadeusiewicz, J.~P�awiak, M.~Sakr, and M.~Hammad, "ECG-COVID: An end-to-end deep model based on electrocardiogram for COVID-19 detection," *Information Sciences*, vol. 619, pp. 324�339, 2023, doi: https://doi.org/10.1016/j.ins.2022.11.069

    %     @article{SAKR2023324,
    % title = {ECG-COVID: An end-to-end deep model based on electrocardiogram for COVID-19 detection},
    % journal = {Information Sciences},
    % volume = {619},
    % pages = {324-339},
    % year = {2023},
    % issn = {0020-0255},
    % doi = {https://doi.org/10.1016/j.ins.2022.11.069},
    % url = {https://www.sciencedirect.com/science/article/pii/S0020025522013585},
    % author = {Ahmed S. Sakr and Pawe� P�awiak and Ryszard Tadeusiewicz and Joanna P�awiak and Mohamed Sakr and Mohamed Hammad},
    % keywords = {COVID-19, ECG, CNN, End-to-end, Deep learning}

    \bibitem{Coutts-HRV}
    L. V. Coutts, D. Plans, A. W. Brown, and J. Collomosse, "Deep learning with wearable based heart rate variability for prediction of mental and general health," Journal of Biomedical Informatics, vol. 112, p. 103610, 2020. doi: 10.1016/j.jbi.2020.103610.

    % @article{COUTTS2020103610,
    % title = {Deep learning with wearable based heart rate variability for prediction of mental and general health},
    % journal = {Journal of Biomedical Informatics},
    % volume = {112},
    % pages = {103610},
    % year = {2020},
    % issn = {1532-0464},
    % doi = {https://doi.org/10.1016/j.jbi.2020.103610},
    % url = {https://www.sciencedirect.com/science/article/pii/S1532046420302380},
    % author = {Louise V. Coutts and David Plans and Alan W. Brown and John Collomosse},
    % keywords = {Machine learning, LSTM, Heart rate variability, Mental health, Wearables},
    % abstract = {The ubiquity and commoditisation of wearable biosensors (fitness bands) has led to a deluge of personal healthcare data, but with limited analytics typically fed back to the user. The feasibility of feeding back more complex, seemingly unrelated measures to users was investigated, by assessing whether increased levels of stress, anxiety and depression (factors known to affect cardiac function) and general health measures could be accurately predicted using heart rate variability (HRV) data from wrist wearables alone. Levels of stress, anxiety, depression and general health were evaluated from subjective questionnaires completed on a weekly or twice-weekly basis by 652 participants. These scores were then converted into binary levels (either above or below a set threshold) for each health measure and used as tags to train Deep Neural Networks (LSTMs) to classify each health measure using HRV data alone. Three data input types were investigated: time domain, frequency domain and typical HRV measures. For mental health measures, classification accuracies of up to 83% and 73% were achieved, with five and two minute HRV data streams respectively, showing improved predictive capability and potential future wearable use for tracking stress and well-being.}
    % }

    \bibitem{andrea-termo}
    Estimation of Heart Rate Variability Parameters by Machine Learning Approaches Applied to Facial Infrared Thermal Imaging
    %https://www.frontiersin.org/journals/cardiovascular-medicine/articles/10.3389/fcvm.2022.893374/full

    \bibitem{Gudny-ppgHRV}
    Gudny Bjork Odinsdottir, Jesper Larsson Deep Learning Approach for Extracting Heart Rate Variability from a Photoplethysmographic Signal

    \bibitem{Morales-RSASVM}
    Morales J, Borz�e P, Testelmans D, Buyse B, Van Huffel S, Varon C. Linear and Non-linear Quantification of the Respiratory Sinus Arrhythmia Using Support Vector Machines. Front Physiol. 2021 Feb 5;12:623781. doi: 10.3389/fphys.2021.623781. PMID: 33633586; PMCID: PMC7901929.
    % https://pmc.ncbi.nlm.nih.gov/articles/PMC7901929/

    \bibitem{Lahr-RSA}
    Peyton Lahr 1,2,Chloe Carling 1,Joseph Nauer 1,Ryan McGrath 1,* andJames W. Grier
    James W. Grier,  Supervised Machine Learning to Examine Factors Associated with Respiratory Sinus Arrhythmias and Ectopic Heart Beats in Adults: A Pilot Study
    % https://www.mdpi.com/2673-3846/5/3/20
    
    \bibitem{Morales_RSASVM_2}
    John Morales, Jonathan Moeyersons, Pablo Armanac, Michele Orini, Luca Faes, Sebastiaan Overeem, Merel Van Gilst, Johannes Van Dijk, Sabine Van Huffel, Raquel Bailon, and Carolina Varon\emph{Model-Based Evaluation of Methods for Respiratory Sinus Arrhythmia Estimation} : DOI 10.1109/TBME.2020.3028204, IEEE
    Transactions on Biomedical Engineering
    % https://iris.unipa.it/retrieve/handle/10447/483248/1119979/112-Morales-IEEE_TBME_2020-preprint.pdf




    \bibitem{TensorFlow-Keras}
    https://www.tensorflow.org/guide/keras
    % https://www.mdpi.com/2673-3846/5/3/20

\end{thebibliography}