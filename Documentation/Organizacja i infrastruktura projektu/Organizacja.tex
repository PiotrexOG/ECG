\documentclass[12pt]{article}
\usepackage{polski}
\usepackage[utf8]{inputenc}
\usepackage{hyperref}
\usepackage{graphicx}

\title{Metody uczenia maszynowego do analizy EKG}
\author{Piotr Wesołowki, Jakub Wojtalewicz}
\date{12.03.2024}
\begin{document}

    \maketitle

    \section{Opis projektu i produktu}

        \subsection*{Nazwa projektu:} 
            Metody uczenia maszynowego do analizy EKG

        \subsection*{Adresowany problem:}
            Automatyczna analiza danych EKG w celu wykrywania określonych wzorców, takich jak HRV i QrsTangle, które mogą być istotne dla diagnozy chorób serca.

        \subsection*{Obszar zastosowania:}
            Medycyna, diagnostyka kardiologiczna, badania naukowe.

        \subsection*{Rynek:}
            Branża medyczna, laboratoria kardiologiczne, instytucje badawcze.

        \subsection*{Interesariusze:}
            Julian Szymański, Politechnika Gdańska, Zespoły medyczne, badacze, producenci oprogramowania medycznego.

        \subsection*{Użytkownicy i ich potrzeby:}
            Lekarze, technicy medyczni, naukowcy potrzebujący narzędzi do szybkiej i dokładnej analizy danych EKG w celu postawienia diagnozy i prowadzenia badań naukowych.

        \subsection*{Cel i zakres produktu:}  
            Opracowanie algorytmu analizy danych EKG z wykorzystaniem sieci neuronowych, umożliwiającego wykrywanie miar HRV oraz bardziej zaawansowanych miar, takich jak QrsTangle. Produkt będzie mógł być wykorzystywany do automatycznej analizy dużych zbiorów danych EKG.  

        \subsection*{Ograniczenia:}  
            Ograniczenia techniczne związane z dokładnością analizy danych EKG, konieczność zapewnienia odpowiedniej jakości oraz ilości danych do treningu sieci neuronowych.  

        \subsection*{Inne współpracujące systemy:}
            Systemy do badania, przesyłania, przechowywania i analizy danych EKG.  

        \subsection*{Termin:}  
            Realizacja projektu przewidziana jest do 31.12.2024 r.

        \subsection*{Główne etapy projektu:}
            \begin{enumerate}
                \item
                Analiza wymagań i specyfikacji produktu.  

                \item
                Opracowanie sposobu przesyłania danych EKG z urządzenia pomiarowego do aplikacji  
                
                \item
                Opracowanie i implementacja algorytmu analizy danych EKG z użyciem sieci neuronowych.   
                
                \item
                Testowanie i walidacja algorytmu na zbiorze danych.  
                
                \item
                Optymalizacja i dostosowanie algorytmu do rzeczywistych zastosowań klinicznych. 
            \end{enumerate}

    \section{Interesariusze i użytkownicy}

        \subsection*{Interesariusze:}
            Zespoły medyczne (lekarze, pielęgniarki), instytucje badawcze (naukowcy, badacze), producenci oprogramowania medycznego.  

        \subsection*{Użytkownicy końcowi:}
            Lekarze, technicy medyczni, naukowcy.  

        \subsection*{Klasyfikacja i krótki opis interesariuszy:}

            \begin{itemize}
                \item
                Zespoły medyczne: Potrzebują narzędzi do szybkiej i dokładnej analizy danych EKG w celu postawienia diagnozy i prowadzenia badań.  

                \item
                Instytucje badawcze: Zainteresowane są możliwościami analizy danych EKG w celu zgłębienia wiedzy na temat chorób serca i wypracowania nowych metod diagnostycznych.  

                \item
                Producenci oprogramowania medycznego: Mogą zainteresować się wdrożeniem algorytmu do swoich
                produktów lub integracją go z istniejącymi rozwiązaniami.
            \end{itemize}

    \section{Zespół}
        
        \subsection*{Kto jest w zespole:}
            \textbf{Piotr Wesołowski}, student 3 roku Informatyki na Politechnice Gdańskiej \\
            \textbf{Jakub Wojtalewicz}, student 3 roku Informatyki na Politechnice Gdańskiej, junior C\#, Unity deweloper

        \subsection*{Umiejętności osób:}
            Doświadczenie oraz bogata wiedza w dziedzinach Inżynierii danych, sztucznej inteligencji, projektowania i tworzenia oprogramowania oraz kardiologii. 

        \subsection*{Obszary odpowiedzialności osób:}
            \textbf{Piotr Wesołowski}: Przetwarzanie i analiza danych EKG, kontakt z interesariuszami \\
            \textbf{Jakub Wojtalewicz}: Opracowanie i dostosowanie modeli sieci neuronowych \\
            \textbf{Zespół}: Implementacja algorytmów analizy danych, projektowanie interfejsu użytkownika, integracja z istniejącymi systemami oraz weryfikowanie wyników analizy 

        \subsection*{Praca w rozproszeniu czy w jednym miejscu:}  
            Praca będzie odbywać się zdalnie i będzie wymagała stałej komunikacji i współpracy między członkami zespołu. 

        \subsection*{Dane kontaktowe osób w zespole:}
            Piotr Wesołowski, student 3 roku, s188923@student.pg.edu.pl, 731791350 \\
            Jakub Wojtalewicz, student 3 roku, s188636@student.pg.edu.pl, 506206963 

    \section{Komunikacja w zespole i z interesariuszami}

        Komunikacja w naszym zespole oraz z naszym opiekunem projektu odbywa się zgodnie z ustalonym harmonogramem i preferencjami. Regularnie utrzymujemy kontakt z opiekunem projektu poprzez wymianę maili oraz uczestnictwo w cotygodniowych spotkaniach, które odbywają się w środy. To pozwala nam na bieżąco omawiać postępy, rozwiązywać ewentualne problemy oraz uzyskiwać cenne wskazówki i opinie od naszego mentora. Natomiast wewnątrz zespołu wykorzystujemy platformy Messenger i Discord do szybkiej komunikacji oraz koordynacji działań. Dzięki temu możemy łatwo dzielić się pomysłami, zadawać pytania i informować o postępie prac, co sprzyja efektywnej współpracy i wspólnemu osiąganiu celów projektu. 

    \section{Współdzielenie dokumentów i kodu }
            
        \subsection*{Repozytorium}
            Projekt "\textbf{ECGAnalysis}" znajduje się na platformie GitHub pod adresem \textit{https://github.com/Ewikk/ECGAnalysis}. Jest ono prywatne i wymaga udzielenia odpowiednich praw dostępowych. 

            Osoba odpowiedzialna za konfigurację i utrzymanie repozytorium: \textbf{Jakub Wojtalewicz} (GitHub użytkownik: Ewikk). 

            Osoba odpowiedzialna za porządek w dokumentacji: \textbf{Piotr Wesołowski} (GitHub użytkownik: PiotrexOG). 


        \subsection*{Sposób wymiany dokumentów i kodu:} 
            Zespół korzysta z repozytorium GitHub do wymiany dokumentów i kodu. Każdy członek zespołu ma dostęp do repozytorium poprzez udzielone uprawnienia. 

        \subsection*{Schemat nazewnictwa dokumentów/plików:}
            Kod źródłowy: \textbf{NazwaPliku.py} \\
            Dokumentacja: \textbf{NazwaDokumentu.pdf, NazwaDokumentu.tex}\\
            Dane: \textbf{NazwaPliku.csv}

        \subsection*{Szablon dokumentu projektu:}
            Szablon dokumentu projektu znajduje się w pliku PROJECT\_TEMPLATE.md w głównym katalogu repozytorium. Zawiera on sekcje dotyczące celu projektu, wymagań, architektury, harmonogramu oraz zadań do wykonania. 

        \subsection*{Sposób wersjonowania dokumentacji:} 
            Dokumentacja jest automatycznie wersjonowana poprzez system kontroli wersji Git na platformie GitHub. Każda zmiana w dokumencie jest odnotowywana poprzez commit z odpowiednim komentarzem opisującym zmianę. 

    \section*{Narzędzia}

        \paragraph*{Wersjonowanie}
            \begin{itemize}
                \item Git
                \item GitHub
            \end{itemize}

        \paragraph*{Komunikacja}
            \begin{itemize}
                \item Discord
                \item Messenger
            \end{itemize}
        
        \paragraph*{Programowanie}
            \begin{itemize}
                \item Visual Studio 
                \item Pycharm
                \item IntelliJ IDEA
                \item Android Studio
                \item Samsung Health SDK
            \end{itemize}

\end{document}